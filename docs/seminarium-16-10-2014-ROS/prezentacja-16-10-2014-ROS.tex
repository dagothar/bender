\documentclass[12pt,a4paper,portrait]{beamer}
\usetheme{default}
\usecolortheme{crane}
\usepackage{polski}
\usepackage[utf8]{inputenc}
\usepackage[polish]{babel}
\usepackage{amsmath}
\usepackage{amsfonts}
\usepackage{amssymb}
\usepackage{graphicx}
\usepackage{ragged2e}
\usepackage{multimedia}
%\usepackage{movie15}
%\usepackage{hyperref}
\usepackage{listings}

\author[AW]{Adam Wolniakowski}
\institute[WM PB]{Politechnika Białostocka}
\title[ROS]{Robot Operating System}
\date{16 października, 2014}
%\logo{\includegraphics[width=1cm]{logo.png}}

\begin{document}

\section{Start}
\begin{frame}
\titlepage
\end{frame}



\section{Wprowadzenie}
\begin{frame}
\frametitle{O czym będę mówić?}
\end{frame}

\begin{frame}
\frametitle{Co to jest ROS?}
ROS (\textit{Robot Operating System}) to open-source'owy, meta-system operacyjny, przeznaczony do zastosowań w robotyce.
Zapewnia to, czego można oczekiwać od systemu operacyjnego:
\begin{itemize}
\item abstrakcję hardware
\item kontrolę urządzeń na niskim poziomie
\item przekazywanie danych między procesami
\item komunikacja między różnymi hostami
\item $\cdots$
\end{itemize}
\end{frame}

\begin{frame}
\frametitle{Pojęcia}

\end{frame}



\section{ROS}
\begin{frame}
\frametitle{Node}
\end{frame}

\begin{frame}
\frametitle{Topic}
\end{frame}



\section{Programowanie}
\begin{frame}
\frametitle{Języki programowania}
\end{frame}

\begin{frame}
\frametitle{Współpraca z Matlabem i Simulinkiem}
\end{frame}

\begin{frame}
\frametitle{PRZYKŁAD}
% tutaj plugin rwsim + wykresy w matlabie
\end{frame}





\section{Bender}
\begin{frame}
\frametitle{Stanowisko 'Bender'}
\end{frame}



\section{Koniec}
\begin{frame}
\frametitle{Koniec}
Dziękuję! Koniec!
\end{frame}

\end{document}